\documentclass[doc 	]{apa}
\usepackage{geometry}        
\usepackage{float}
\geometry{a4paper}          
\usepackage{graphicx}
\usepackage{amssymb}
\usepackage{amsmath}
\usepackage{apacite}
\usepackage{multirow}
\usepackage{color}
\usepackage{enumitem}
%\usepackage{multicol}
\usepackage{longtable}
\usepackage{pdflscape}
\usepackage{caption}
\usepackage{subcaption}
\usepackage{mdframed}
\captionsetup{compatibility=false}
\captionsetup[table]{labelsep=space}

\acknowledgements{Correspondence concerning this article may be addressed to: Scott Brown, School of Psychology, University of Newcastle, Callaghan NSW 2308, Australia; Email: scott.brown@newcastle.edu.au}
\affiliation{
$^a$ Department of Psychology, University of Amsterdam, The Netherlands  \\
$^b$ School of Psychology, University of Newcastle, Australia  
}

\shorttitle{Linking behavioural and neural data}
\rightheader{Linking behavioural and neural data}
\leftheader{Linking behavioural and neural data}

\begin{document}
\title{Different ways of linking behavioral and neural data via computational cognitive models}
\author{Gilles de Hollander$^a$, Birte U. Forstmann$^a$, and Scott D. Brown$^b$}


\maketitle

\tableofcontents

\section{Abstract}
Cognitive neuroscientists sometimes apply formal models to investigate how the brain implements cognitive processes. These models describe behavioral data in terms of underlying, latent, variables linked to hypothesized cognitive processes. A goal of model-based cognitive neuroscience is to link these variables to brain measurements, which can advance progress in both cognitive and neuroscientific research. However, the details and the philosophical approach for this linking problem can vary greatly. We propose a continuum of approaches which differ in the degree of tight, quantitative, and explicit hypothesizing. We describe this continuum using four points along it, which we dub ``qualitative structural'', ``qualitative predictive'', ``quantitative predictive'', and ``single model'' linking approaches. We further illustrate by providing examples from three research fields (decision making, reinforcement learning, and symbolic reasoning) for the different linking approaches.

\section{Introduction}
In recent years, cognitive neuroscientists have applied formal, computational
cognitive models to more effectively understand how the brain implements cognitive processes such as decision making, reinforcement learning, and symbolic reasoning. Such formal cognitive models can
decompose effects in behavioral data by description in terms of underlying latent cognitive processes and
associated variables. ``Model-based cognitive neuroscience'' links these
variables to brain measurements. This approach can, on the one hand, constrain
the development of cognitive models, while, on the the other hand, also refine
models that explain how cognitive processes are implemented in the brain
\cite{forstmann2011reciprocal}. 

Linking brain measurements to psychological constructs has been conceptualized as identifying a\emph{bridge locus} \cite{Teller:1984vj, Schall:2004ei}. \citeA[p. 1240]{Teller:1984vj} defines the
simplest criterion for identifying a bridge locus as follows: ``if psychophysical and physiological data can be plotted on meaningfully similar
axes, such that the two graphs have similar shapes, then that physiological
phenomenon is a major causal factor in producing that psychophysical
phenomenon''. Formal, computational cognitive models offer a means to make these axes more
meaningful. But how do we link cognitive models to functional brain
measurements most effectively? In the past decade, parameters of formal
cognitive models have been linked to many measures of neural activity, such as
EEG, fMRI, and single-cell recordings. These studies have employed wildly
varying approaches, connecting variability in behavior and brain measurement at
the level of subjects, conditions, and even trials. In some studies, cognitive
models are used to set up testable hypotheses about brain activity. In other
studies, cognitive model parameters are directly correlated against measurement
models of neural data, after both models have been fit to their respective data
domain. Some studies make a single model of both brain and behavior and try to
predict both at the same time.

In this review we aim to give a wide overview of what kind of linking has been
used in the empirical literature, propose one particular taxonomy of these
methods, and lay out future challenges and developments.

\begin{landscape}
\begin{center}
\begin{longtable}{ | p{4cm} | p{5cm} | p{5cm} | p{5cm} |}
    \hline
     %% GILLES TO SCOTT: just out of curiosity, do you mean 'paradigm' as "Random dot motion paradigm" or in the Popperian sense?
     %% SCOTT TO GILLES: I meant it in the sense of *experimental* paradigm. But I take your point that it's unclear, and have removed the reference.
      & \emph{Evidence accumulation} & \emph{Reinforcement learning} & \emph{Symbolic reasoning} \\ \hline
     \\
     \textbf{Linking type}
     \\ \hline
     \emph{Qualitative structural linking} &
 	\textbf{\citeA{UsherMcClelland2001}}: the leaky competing accumulator model assumes that in evidence accumulation, accumulators corresponding to different choice options inhibit each other and also passively decay. This is inspired by the neural observations of mutual inhibition between neurons and passive decay of membrane potentials.
	&
	\textbf{\citeA{Frank:2006bf}}: a neural network model simulates dopamenergic basal-ganglia-cortical interactions. This is a neurocomputational model that makes qualitative predictions about both brain activity and behavior.
	&
	\textbf{\citeA{anderson:2007}}: the general ACT-R framework has, in its later versions, been explicitly linked to fMRI brain research. The framework assumes distinct cognitive modules and relates them tto different parts of the brain.
	\\
    \hline
     \emph{Qualitative predictive linking} &
 	\textbf{\citeA{Ho:2009da}}: evidence accumulation models predict distinct fMRI time courses for different drift rates. This qualitative prediction was supported by differences in the HRF across the entire brain. &
	\textbf{\citeA{Berns:2001uo}}: neural recordings from two conditions with differing predictability in upcoming response with a certain associated value, as estimated by a temporal difference learning model are contrasted.
	&
	\textbf{\citeA{Borst:2010de}}: a priori predictions about differences across conditions in cognitive modules predict cortical activations, that are tested using fMRI. \\
	\hline
	\emph{Quantitative predictive linking} &
 	\textbf{\citeA{van2011neural}}: trial-to-trial fluctuations in response caution as estimated by STLBA are correlated with single-trial BOLD estimates. &
	\textbf{\citeA{ODoherty:2003wy}}: an error signal as estimated by a reinforcement learning model was regressed against BOLD activity. OFC and ventral striatum coded for this signal.
	&
	\textbf{\citeA{Borst:2011jl}}: An ACT-R model predicted activity patterns for different cognitive modules during a multitasking paradigm. The activity of these modules was convolved with an HRF and regressed against fMRI volumes, to identify brain regions corresponding to the different cognitive modules.
	 \\
    \hline
	\emph{Single model} &
 	\textbf{\citeA{Purcell:2012kr}}: constructed a model that takes raw neural data of a visual area as input. This model can then predict distributions of both behavioral data as well as neural data in a downstream evidence accumulation area &
	&
	\textbf{\citeA{anderson2011tracking}}: a simple RT-model of cognition was linked to a multivariate model of neural activation in fMRI by a Hidden semi-Markov model. The resulting joint model made predictions about cognitive states and neural recordings.
	\\
\hline
\caption{Examples of four different approaches to linking models of evidence accumulation, reinforcement learning, and symbolic reasoning to neural data.} \label{table:examples}
\end{longtable} \end{center}
\end{landscape}

\section{Looser and Tighter Links}

There are many approaches to linking formal models of cognition to neural data.
These approaches differ in how explicit and precise the link is made between
neural, physiological processes on the one hand, and cognitive, phenomenal
processes on the other hand. We propose a continuum of ``�tightness'' of
linking��. At the loosest level, cognitive models can be linked with
neural data simply by constraining the kinds of structural assumptions allowed
in the models in order to respect data about neural structures. Tighter links
can be created by comparison of predictions for neural and behavioral data, or
neural and behavioral model parameters. The very tightest and most explicit
links are specified by ``��joint'' models�� which make quantitative
predictions about both neural and behavioral data at the same time. Table
\ref{table:examples} provides some illustrative examples which are elaborated
below. These examples highlight four commonly-used categories on the continuum
between loose to tight linking.

Below, we first provide definitions for those four different commonly-used levels of linking. Following that, we give detailed examples of these approaches in practice, with each level of linking illustrated in up to three different research domains: perceptual decision-making; reinforcement learning, and symbolic reasoning.

\begin{description}
\item \emph{Qualitative structural linking}: Neural data on the structure of
the brain are used to constrain the structure of a cognitive model. An example
of this is the leaky competing accumulator model (LCA): ``the principles
included in the modelling effort have neurobiological as well as computational
or psychological motivation, and the specific instantiations of the principles
are informed by additional neurophysiological observations'' \cite<p.
554;>{UsherMcClelland2001}.
\item \emph{Qualitative predictive linking}: A cognitive model is tested using
qualitative predictions about both neural and behavioral data. For example,
\citeauthor{Borst:2010de} used the symbolic reasoning modelling framework of
ACT-R to make predictions about the difference in fMRI-signals between conditions which differed in behavioral measures associated with task difficulty, separately for different brain regions: ``the model does not predict a general
increase in BOLD response with task difficulty; instead, it predicts lower but
more persistent activation levels for the more difficult conditions in the
visual and manual modules, and higher and more persistent activation levels for
the more difficult conditions in the problem state and declarative memory
modules'' \cite<p. 6;>{Borst:2010de}.
\item \emph{Quantitative predictive linking}: the predictive output of
a cognitive model is quantitatively related to some aspect of neural data. In an early example of this approach, fMRI data was acquired during a Pavlovian
conditioning task. The signal that was measured by fMRI was correlated with the
error signal of a temporal difference (TD) algorithm performing the same task:
``we used the actual output of a TD learning algorithm to generate a PE (or
$\delta$) response at two main time points in a conditioning trial: the time of
presentation of the CS and the time of presentation of the reward. The output
of this algorithm was then entered into a regression model of fMRI measurements
from subjects who underwent appetitive Pavlovian conditioning. This enabled us
to test for brain regions that manifested a full range of TD error-related PE
response'' \cite<p. 330;>{ODoherty:2003wy}.
\item \emph{Single model}: a single generative model predicts a joint
distribution over both cognitive and neural data. For example, \citeauthor{Purcell:2010jo} used single-cell recordings from monkeys:
``Models using actual visual neuron activity as input predicted not only the
variability in observed behavior but also the dynamics of movement neuron
activity. This union of cognitive modeling and neurophysiology strengthens the
interpretation of visual neuron activity as a representation of perceptual
evidence of saccade target location and the interpretation of movement neuron
activity as the accumulation of that evidence'' \cite<p. 30;>{Purcell:2010jo}.
\end{description}

We will now present the four levels of linking using examples from the fields of evidence accumulation models, value-based decision-making models, and symbolic reasoning models.

\section{Examples of qualitative structural linking}

\subsection{Qualitative structural linking in models of evidence accumulation}

Evidence accumulation models have been used to explain simple decision-making processes for more than fifty years \cite{Stone1960,Ratcliff1978,Luce1986}. More recently, attempts have been made to link the models with neural data. The earliest attempts, such as seminal work by \citeA{UsherMcClelland2001}, defined qualitative structural links. These links were structural in the sense that the constraints were applied to the structure of the model, not to the model's predictions, and the links were qualitative in the sense that the constraints revolved around the inclusion/exclusion of model elements, not to the quantitative parametric values taken.

For example, the leaky competing accumulator model (LCA) of \citeA{UsherMcClelland2001} specifically included structural elements such as mutual inhibition between competing accumulators. This inclusion was motivated by neural data which demonstrate the prevalence of inhibitory connections between nearby neurons within the same cortical stratum. Similarly, the LCA included passive decay of accumulated evidence, to respect the neural observation that membrane potential decays back to baseline in the absence of input. Evidence in favor of these links was inferred by the observation that the resulting cognitive model provided a good fit to behavioral data.

\citeA{Smith:2010ht} showed that a plausible model of how neurons encode
sensory information at very short time scales (a Poisson shot noise process),
converges, with some assumptions, to a Ornstein-Uhlenbeck velocity process.
The integrated version of this process is, in turn, indistinguishable from
a Wiener process and the Wiener process is the diffusion process that is
assumed to underlie the standard drift diffusion model \cite{Ratcliff1978,
Ratcliff:2008fz}. Smith thus showed that the DDM is a suitable abstract model of the neural dynamics of decision-making at larger time scales.

\citeA{Ratcliff:2012je} showed, in a similar vein, that simulated ``behavioral
data'' produced by a prominent neurocomputational model of perceptual
decision-making in corticostriatal circuits \cite{Frank:2006iz} are
well-explained by the DDM. Importantly, parametric changes in projection
strengths in the neurocomputational model (those between the subthalamic
nucleus and globus pallidus, internal segment) were reflected by corresponding
parametric increases in decision threshold and non-decision time during
high-conflict decisions when the parameters of the DDM were estimated on the
behavioral output of the neurocomputational model.

\subsection{Qualitative structural linking in models of reinforcement learning}

The classic parallel distributed processing models provided cognitive descriptions of learning including structural constraints from neural data \cite<PDP:>{RumelhartHintonMcClelland1986,RumelhartHintonWilliams1986}. The models assumed massive parallelism and distributed information representation, reflecting key findings in the emerging neural literature on cortical structure. The models also used learning rules such as back-propogation, which were inspired by neural findings such as Hebbian plasticity. 

An important contribution of the PDP approach was its demonstration that very simple structures -- such as those found in cortex -- were sufficient to support quite complex computations, when endowed with the appropriate representation and learning assumptions. This contribution emerged directly out of the effort to draw structural links between neural and cognitive data.

\subsection{Qualitative structural linking in models of symbolic reasoning}

The ACT-R production framework \cite{Anderson1992} is a domain-general model of human cognition. This model began as a cognitive model purely aimed at behavioral data, but has since been extended in great detail to jointly consider behavioral and neural data \cite<e.g.,>{Sohnetal2003,Qinetal2003}. The earliest linking of the ACT-R model to neural data was qualitative structural linking, which identified links between different cognitive modules in ACT-R and different brain regions. These links respected findings about the localization of brain function that were emerging at the time from the then-new method of fMRI. For example, the ``visual module'' of ACT-R was linked with lower occipital brain regions, and the ``motor module'' with motor cotices in the parietal and temporal lobes. These links defined the structure of the model and allowed the investigation of hypotheses about deficits due to brain lesions, for example. 


\section{Examples of qualitative predictive linking} 

One could argue that almost all work in cognitive neuroscience, and particularly cognitive neuroimaging, belongs in this category, because this research
usually assumes at least some cognitive concepts like``working memory'',
``attention'', or ``semantic memory'' \cite{gazzaniga2007cognitive}, which, taken
together, could be interpreted as a model. This cognitive model is then
compared to neural data from an experimental paradigm that modulates the
cognitive concept of interest. To be more specific, we will only focus on the subclass of
computational cognitive models that allow for generative, quantitative predictions
\cite{lewandowsky2010computational}. We make this restriction because these kinds of models
also offer the possibility of tighter, quantitative links to neural data,
discussed in later sections.

\subsection{Qualitative predictive linking in models of evidence accumulation}

\citeA{Hanes:1996ww} recorded single-cell activity
in the frontal eye fields (FEF) in behaving macaques. The activity of
 ``movement neurons'' predicted the execution
of saccades. \citeA{Hanes:1996ww} showed that the ramping activity of these
neurons preceding a saccade always ended with the same firing rate, but the
rate of increase of firing rate was variable. The authors related these
qualitative patterns to evidence accumulation models. In certain evidence accumulation models, evidence builds up gradually before a response is made, with two key properties: the rate of build-up (the ``drift rate'') differs from decision to decision, but the amount of accumulated activity just before a response is issued (the ``threshold'') is always the same. \citeA{Hanes:1996ww} interpreted their findings as showing that variability in response time could be explained by variability in drift rate as
opposed to variability in threshold of the decision-making process, a claim
that is hard to test when using only behavioral data. More and
more electrophysiological work has since been interpreted in the framework
offered by evidence accumulation models, reviewed by \citeA{Gold:2001vu} and \citeA{Forstmann:VLiQuR2S}.
%% GILLES to SCOTT: this is a review paper by Birte, Roger, and EJ that will be published in annual review of psychology
%% SCOTT to GILLES: Excellent. I don't know that paper, but please add it if you see fit.

\citeA{Churchland:2008ds} noted that some evidence accumulation models also predict
increased decision thresholds when the number of choice alternatives increases. This change in threshold is required in some models to counter the effect of ``statistical facilitation''; the tendency for the fastest-finishing of a set of accumulators to become even faster as the set grows larger \cite<e.g.>{UsherEtAl2002}. Increased thresholds counter both the decreased RT as well as the increased error rates associated with statistical facilitation, and bring model predictions into line with behavioral data, such as Hick's Law. 

From a neural perspective, increased decision thresholds can be implemented in one of two ways: either by increasing the firing rate required to trigger a behavioral response, or by decreasing the baseline firing rate before a decision. \citeA{Churchland:2008ds} investigated these two possibilities, by examination of  neurons in the lateral intraparietal area (LIP) which behaved like evidence
accumulators. Those neurons showed reduced baseline activity when more response choices were added. The subtle
differences between a reduced baseline or an increased threshold are something
that can not be distinguished using behavioral data only \cite{Smith:2004jo}.

Qualitative links between neural data and evidence
accumulation models have also been drawn using fMRI methods. For example, \citeA{Ho:2009da} hypothesized that areas that implement evidence accumulation
during a perceptual decision-making task should show a delayed, temporally
extended hemodynamic response function (HRF) during difficult trials, as compared to
easy trials. They identified areas where the shape of the HRF differed
substantially between conditions, by testing for interactions between task
difficulty and BOLD activity at a set of multiple timepoints throughout the
trial. The raw signal in these areas was averaged over trials, and this indeed
showed the predicted qualiative pattern of delayed and longer hemodynamic
responses for trials from a hard condition, as compared to the easy condition.

\subsection{Qualitative predictive linking in models of reinforcement learning}

The field of reinforcement learning and value-based decision-making has a long
history of computational cognitive modelling \cite{Sutton:1998:IRL:551283}.
These computational models made it possible to design experiments that
manipulated model parameters across conditions and compare the corresponding
neural and behavioral data \cite<e.g.,>{Berns:2001uo,Knutson:2001vi}.
\citeA{Nieuwenhuis:2002uk} showed that modulating a single parameter in Holroyd
\& Coles' neurocomputational reinforcement learning model could explain
differences between age groups, by assuming plausible neuroanatomical
differences. Nieuwenhuis et al. observed that the Holroyd and Coles' model
could mirror the the impaired performance of older adults in a probabilistic
learning task, as well as the accompanying reduced error-related negativity
(ERN) measured by EEG. It could do so by varying only one parameter in the
model that represents to the efficiency of dopaminergic connections to the
anterior cingulate cortex (ACC). 

Since the early 2000s, the estimated parameters of reinforcement learning
models have been quantitatively compared to differences in the neural signal, leading
to a small revolution in the field, which will be discussed in the next
section on quantitative predictive linking.

\subsection{Qualitative predictive linking in models of symbolic reasoning}

Continuing with our previous example, the ACT-R model assumes distinct
cognitive modules that perform different parts of cognitive tasks
\cite{anderson:2007}. For example, the cognitive steps necessary for
performing some symbolic logic operation might be modeled as involving the
visual module (to perceive the stimulus), the procedural and declarative
memory modules (to remember the logical rules), and the motor module (to
produce the desired behavioral response). From these assumptions, ACT-R can
make predictions about differences in reaction time and accuracy between
conditions. Many neuroimaging studies have related cognitive ACT-R models to
fMRI data to localize the cognitive modules within the brain. Such
localization assumptions are linking hypotheses, and subsequent studies
have used qualitative predictive approaches to test those. For example,
\citeA{Borst:2010de} constructed an ACT-R model of a task where both
a subtraction operation, as well as a text entry had to be performed at the
same time. The model made \emph{a priori} predictions about which modules
(e.g., ``Problem State'', ``Declarative Memory'', ``Manual'', and ``Visual'')
would be more activated during different combinations of easy/hard versions
of the two tasks. These predictions about cognitive modules can be manifested
as predictions about the behavioral data (via ACT-R's regular framework) and
also as corresponding predictions for fMRI data, via the linking hypotheses
which localize modules. These predictions were confirmed by
\citeA{Borst:2010de}, using a traditional region-of-interest fMRI study with
different task contrasts.

The ACT-R community has come up with a ``standard atlas'', predicting where
in the brain BOLD activity should be expected, given hypothetical activity of
cognitive modules \cite{Anderson:2008ea,Borst:2015et}. The latest version of
ACT-R software can even, given a model of a cognitive task, predict
qualitative differences in BOLD-activity across different versions of a task.
In recent work, these predictions have been made quantitative and can be
empirically tested in fMRI data. This will be discussed in the next section.

%This still uses qualitative constraints, but now applies them to the predictions of the model. So a model is used to generate qualitative predictions for neural data and corresponding behavioural data. E.g. In the ACT-R model, better recall in experimental condition A than B, and correspondingly larger BOLD signal in condition A than B from brain regions corresponding to the memory module of ACT-R (I'm a bit foggy on the details of this -- one of us will need to read up on some better ACT-R examples, or other models).

%There are lots of examples in decision-making. The cortico-striatal model of SAT predicts greater activity in basal ganglia input regions in speed than accuracy conditions, and corresponding faster RTs in speed than accuracy conditions \cite{ForstmannEtAl2008}. Linking a region of the brain with evidence accumulation predicts faster onset of BOLD activation in easy than difficult conditions, and correspondingly faster RTs in easy than difficult conditions \cite{HoBrownSerences2009}. In monkeys, linking LIP or FEF neurons with accumulation dynamics predicts faster increase from baseline to threshold firing rate in easy than difficult conditions, and correspondingly faster RTs \cite{RoitmanShadlen2002,Schall2003}.

%Maybe some of Philip and Roger's work fits in here. The shot-noise Poisson stuff, and the work with Philiastedes.


\section*{Box 1: single-trial regression approach in fMRI}
\renewcommand{\figurename}{Box}
\begin{figure}[H]
\caption*{The approach to linking fMRI measurements to cognitive models which currently dominates cognitive model-based neuroimaing is the so-called
\emph{single-trial regression approach}. To understand how the single-trial
regression approach works, it is helpful to first have a look at the related,
traditional analysis of fMRI data using the general linear model (GLM).}
\begin{subfigure}{0.85\textwidth}
  \centering 
  \includegraphics[width=\linewidth]{figure2a.eps}
  \caption{\emph{The traditional GLM-approach} 
	\\\hspace{\textwidth} The traditional GLM approach aims to
	find differences in BOLD activation between experimental conditions.
	A hypothetical neural signal is constructed for every
	condition in the experimental design using block functions (top
	panel). These hypothethical signals are then convolved with
	a canonical HRF to make them resemble fMRI data. These hypothetical
	BOLD responses are used as regressors $X_i$ in a GLM (middle panel).
	The parameters of the GLM are estimated as to construct a weighted
	sum of the regressors that most resembles the real BOLD signal. This
	yields a so-called $\beta$-parameter estimate for every voxel and
	every condition (bottom panel). This parameter can be interpreted as
	the estimated height of the BOLD-response for that particular
	condition in that particular voxel. By subtracting beta-values and combining their
	variances, researchers can infer which voxels show a significant
	difference in BOLD-response across conditions. The large majority of
	fMRI-studies employ this analysis strategy.
	} \label{fig:glm_approach}
\end{subfigure}
\end{figure}

\setcounter{figure}{0}
\begin{figure}[H]
\begin{subfigure}[b]{0.85\textwidth}
  \setcounter{subfigure}{1} \centering
  \includegraphics[width=\linewidth]{figure2b.eps} 
  
  \caption{\emph{the single trial
  regression approach} \\\hspace{\textwidth} The traditional GLM approach can be
  extended to the single trial regression approach. Instead of estimating
  the heights of the BOLD response in different conditions, one estimates the correlation across trials of the BOLD signal with some hypothetical
  '`cognitive'' signal. In model-based neuroimaing, this signal is usually (a
  transformation of) a parameter of the cognitive model that can be estimated
  for single trials. These parameters are converted to a hypothetical
  neural signal by multiplying them with a block function, and this block
  function is then converted to a hypothetical BOLD response by convolution
  with a canonical HRF. \\\hspace{\textwidth} The parameters of this
  model-based GLM can also be fitted to every voxel in the brain, resulting in
  a ``correlation map''. Significant voxels in this map correspond to brain areas
  in which the BOLD response is reliably associated with the cognitive model
  parameter-of-interest. \\\hspace{\textwidth} The main advantage of the
  single-trial-regression approach is its ease-of-use. The main fMRI software
  packages (SPM, FSL, AFNI...) support the regression-approach out-of-the-box
  and all a neuroimager needs to get started is one value per trial to put into
  her GLM. \\\hspace{\textwidth}The approach also has some disadvantages.
  Firstly, for some cognitive models it is not feasible to get single-trial
  estimates of a particular parameter. For example, evidence
  accumulation-models are fitted to an entire distribution of
  behavioral measurements and there exist no parameter estimates for single
  trials \cite<but see>{van2011neural}. Secondly, in the conventional
  regression approach, the parameters of the cognitive model and the
  measurement model are estimated separately. Estimating them together in
  a single model can increase the precision of the estimates
  \cite{Turner:2010kt}. Lastly, the regression approach has very strict
  linking assumptions. Researchers using it have to assume that across the
  entire brain and every subject, the task-related BOLD response follow the
  canonical HRF, and this response is modulated by the cognitive parameter
  only in its height, in a linear fashion. In the traditional single-trial
  regression approach it is virtually impossible to relax these assumptions.}

\end{subfigure}
\end{figure}

%\end{mdframed}


\section{Examples of quantitative predictive linking}

Computational cognitive models can also be linked to neural data in
a quantitative way. This approach involves two models which are combined:
a cognitive model that predicts behavior, as well as a neural model that
predicts neural signals. Often, one of these two models is very much
simplified. For example, the cognitive model may be as basic as simple signal
detection theory, or the neural model may be as atheoretical as the mean signal
between two response-locked timepoints. 


\subsection{Quantitative predictive linking in models of reinforcement learning}

Using quantitative outputs of a computational model of cognition to predict
neural activity has been a successful strategy in the study of value-based
decision-making and neuroeconomics \cite{Corrado:2009uj}. Especially
prominent has been the \emph{single-trial regression approach}, in which
parameters of a reinforcement learning model are estimated from choice
behavior during tasks involving the learning of reward values associated with
different choices. These subject-specific parameter estimates can be used to
calculate estimates of the subjective values of the different choice options
to the subject, for every individual trial during the experiment (note that
these values are dynamic due to learning). These subjective, trial-by-trial
values can then be used as a hypothetical cognitive signal that tracks, for
example, the difference between the expected reward after a choice and the
reward that was actually delivered (the so-called ``prediction error'' or
``delta'' signal). To investigate a linking hypothesis, the researcher then
hypothesizes that this cognitive signal is represented in the brain, at the
\emph{bridge locus}. Neural signals from the bridge locus should correspond
to the phenomelogical concept under study, and to the hypothetical cognitive
signal in this case \cite{Teller:1984vj}. For example, the bridge locus of
the prediction error signal might be some area in the brain where the neural
signal consistently tracks the difference between the expected reward and the
actual reward in a reinforcement learning task. At a practical level, the
hypothetical cognitive signal, as estimated by the reinforcement learning
model, can be transformed to a hypothetical BOLD fMRI-signal, by convolution
with a hemodynamic response function
\cite{Glover:1999vv,friston2011statistical}. This creates a hypothetical fMRI
signal corresponding to the prediction error signal, which can be used as
a regressor in a general linear model (GLM), with additional regressors for
other task-related activity (for example stimulus presentation). The
parameters of this GLM are then estimated for all voxels in the brain. This
yields a statistical parametric mapping of the brain that shows for which
areas of the brain, BOLD activity correlates with the hypothetical neural
signal representing stimulus value, and offers candidates for the bridge
locus of interest (see Box 1 for a more detailed explanation of the
single-trial regression approach and illustrations). 


%% SCOTT to GILLES: Is O'Doherty's study really "the first"? That's a big claim! If you aren't quite sure, maybe change to "an early"?
%% I'm pretty sure it was the first.
%% Birte said I should cite Daw. I think she means Daw, N. D., O�Doherty, J. P., Dayan, P., Seymour, B., & Dolan, R. J. (2006). Cortical substrates for exploratory decisions in humans. Nature, 441(7095), 876�879. doi:10.1038/nature04766 

% However, this study is 3 years after o' Doherty, and the main finding is actually a contrast over 2 conditions: explorative versus exploiting trials, as estimated by a reinforcement learning model. So it's _not_ a regression approach.
The first studies in cognitive model-based fMRI studies that used the
regression approach were in the field of reinforcement-learning. For example,
the approach was used to show that the BOLD activity in orbitofrontral cortex
(OFC) and in ventral striatum was correlated with the temporal difference error
signal as estimated by a reinforcement learning model \cite{ODoherty:2003wy}.
Many studies using this approach were then published in the domain of
reinforcement learning. It has, for example, been shown that different parts of
the striatum code for different kinds of value representations
\cite<representing the value of states versus the value of
actions;>{ODoherty:2004kcb} and that uncertainty about the stability of the
payoff structure of the environment is computed and learning rate accordingly
modulated by the anterior cingulate cortex \cite{Behrens:2007kk}.
\citeA{Rodriguez:2006kv} applied the approach to the Rescorla-Wagner model
during a categorization task, without explicit reward or punishment. Also more
neurocomputational models, predicting how the different cortical inputs to
striatum are integrated have been tested \cite{Haruno:2006gd}. 

A recent study by \citeA{denOuden:2009gl} extended the single-trial
regression approach to link parameter estimates of a reinforcement learning
model to trial-to-trial variability in not only the \emph{magnitude of
activation} in brain areas, but also to the \emph{functional connectivity}
between brain areas. The cognitive model was used to predict on which trials
activity between two areas was more correlated. The authors applied the
Rescorla-Wagner model, a simple reinforcement learning model, to a paradigm
where auditory stimuli -- unrelated to the task at hand -- were predictive of
visual stimuli that were also unrelated to the task. Because this relation
was implicit, no behavior was available to fit the RL model to, but it was
shown that functional connectivity between auditory and visual cortex was
modulated by the prediction error. The authors used the single-trial
prediction error, estimated by the RL model, as input for a dynamic causal
modelling analysis \cite<DCM;>{Friston:2003cv}. Dynamic causal modelling can
be used to study the causal influence of multiple brain areas on each other's
BOLD signal. A DCM that assumed that the influence of the BOLD signal in
auditory cortex on the BOLD signal in visual cortex was modulated by the
prediction error of a trial was more likely than other models. This finding
suggests that effective connectivity between sensory areas is increased when
the degree of surprise in the environment is larger, with the direction of
causality corresponding to the direction of information flow. 

A follow-up study \cite{denOuden:2010co} used a perceptual decision-making
task (faces vs. houses) with cues that predicted upcoming stimuli. DCM showed
that BOLD activity in the ventral striatum, related to prediction error (as
quantified by a reinforcement learning model) modulated the functional
connectivity between visual areas and motor areas. This result provides
evidence that the detection of uncertainty in the environment increases the
the neural influence of perceptual areas on motor areas.   

\subsection{Quantitative predictive linking in models of symbolic reasoning}

%The ACT-R community constructs higher-level models of cognition that assume
%distinct cognitive modules and non-overlapping sequences of steps during information processing.
%% SCOTT to GILLES: I've commented out the paragraph below, because I'm not convinced it adds enough to warrant the thematic detour.
%\citeA{anderson2011tracking} extended these results by constructing multiple
%versions of a hidden markov model for algebra tasks, with a varying number of
%discrete substates within solving a problem. Model comparison techniques can
%then help to find out how many discrete brain states exist within the process
%of solving an algebra problem, taking into account both neural data, in the
%form of fMRI activation volumes, as well as behavioral data, in the form of
%reaction times.

The single-trial regression approach has also gained a firm hold in the field
of symbolic reasoning models. Recent versions of the
ACT-R architecture predict quantitative differences in activation in
different cognitive modules during a task. These predictions can be convolved
with a canonical hemodynamic response function, generating quantitative
hypotheses about which areas of the brain modulate their activity in
correspondence with the activity of the proposed cognitive modules in the
model. The first study that showed the potential of this approach used
a complicated multitasking paradigm, where either a subtraction or a text
entry task had to be performed, while at the same time performing a listening
comprehension task \cite{Borst:2011jl}. The ACT-R model could predict, for
every trial, the relative activity of ``Problem State'', ``Declarative
Memory'', ``Vision'', and ``Manual'' modules. Single-trial regression
analysis showed correlated activity in corresponding cortical areas, largely
overlapping with areas that had been related to these modules before by
\citeA{anderson:2007}.
% Interestingly, the
%analysis found very similar brain areas that correlated to ``Vision'' and
%``Manual'' modules, namely a mix of both primary visual and primary motor
%areas. This lack of dissociation most likely arose because perceptual and
%motor activity always occured at the same time. This made it impossible to
%dissociate the two and this warns us that also in model-based neuroimaging, an
%experimental design that orthogonalizes dependent variables of interest is
%crucial.
%% SCOTT to GILLES: As before, I suspect that the commented-out bit above is a level of detail that we don't really need.

\citeA{Borst:2013fm} extended the results of \citeA{Borst:2011jl}, by applying
the same approach to five previously acquired datasets and combining the
results into a meta-analysis. Declarative memory retrieval, as predicted by the
ACT-R models, correlated with activity in the IFG and the ACC, whereas updating
of working memory was also related to activity in the inferior parietal lobule
(IFL). The authors claimed that model-based imaging can help elucidate the
roles of the different nodes in the frontoparietal network. This network
includes both the IFG, ACC, as well as IFL and dissociating the individual
roles of these nodes has turned out to be challenging when using the
conventional subtraction-of-conditions approach, because the activity of the
nodes in this network are highly correlated.

% GILLES TO SCOTT: I used 'harder', because I think it is harder: due to the stochatic nature of evidence accumulation models, there are no nicely defined 'single trial estimates' of their parameters. I'm not sure anymore whether this should go in the subsection title, but I do think it's useful to reflect a bit on this, as this is a review paper...
% SCOTT to GILLES: I see your point. I've removed ``harder'' from the heading, because I've also tried to make the headings consistent throughout the document. I will try to add some reflection on the difficulty of the process to the text.

%\subsection{The link becomes harder: evidence accumulation models}
\subsection{Quantitative predictive linking in models of evidence accumulation}

Linking evidence accumulation models of speeded decision-making to neuroimaging
data represents a more difficult challenge, in many ways, than for the models
of reinforcement learning and symbolic reasoning reviewed above. The challenge
is more difficult because models of speeded decision-making have to explain
variability in reaction times, and the processes underlying decision-making are
therefore assumed to be stochastic. Across trials, there is variability in the
amount of evidence that is necessary to make a decision, as well as variability
in the average speed of evidence accumulation \cite<as in the
LBA;>{brown2008simplest}, possibly amongst even more variability \cite<e.g.,
variability in non-decision time DDM;>{Ratcliff:2008fz}. This means that,
unlike in most reinforcement learning models, there is no one-to-one
correspondence between data and the parameters of the model at the level of
a single trial. This precludes the very popular single trial regression
approach, at least for ``out-of-the-box'' evidence accumulation models.
Different alternatives to resolve this issue have been proposed and performed.

One alternative is to change the unit of analysis from single trials to single
subjects, focusing on the covariance of differences between subjects in neural
and behavioral parameter estimates. In an fMRI study of decision-making,
\citeA{forstmann2008striatum} instructed subjects to stress either the speed or
accuracy of their decisions. The difference in BOLD-activity between accuracy-
and speed-stressed trials in the striatum and the presupplementary motor area
(pre-SMA) was correlated across subjects with the difference in model
parameters related to response caution, estimated from behavioral data via the
LBA model. In other words, participants who made large changes in their
cognitive settings (for speed vs. caution) also
showed large changes in fMRI responses, and vice versa. This provides some
evidence for a role of these brain areas in setting a response threshold before
a decision is made. In a follow-up study, \citeA{forstmann2010cortico} extended
the earlier finding to structural connectivity measures. They identified
a correlation between individual differences in response caution between speed-
and accuracy-stressed trials and the strength of white matter connections
between pre-SMA and striatum as measured by diffusion weighted imaging (DWI).
The study showed evidence that the stronger these connections are, the more
flexible subjects are in adjusting their response threshold to the current task
demands. 

Using a similar across-subjects approach, \citeA{Mulder:2012gh} used probabilistic payoffs to shift the decision biases of participants. As usual, these shifts were explained in a perceptual decision making model (the drift diffusion model) as a shift in the starting point parameter -- responses favored by bias were represented as having starting points for evidence accumulation that were closer to the response threshold. Mulder et al. showed that estimates of the start point, taken from behavioral data, were correlated with the difference in fMRI
activity between biased and unbiased trials in frontoparietal regions involved
in action preparation.

An alternative to the between-subjects approach is to link within-subject variability from neural and behavioral
data by splitting the data on a neural measure and fitting a cognitive
model to the subsets of behavioral data. \citeA{Ratcliff:2009ei} studied a perceptual decision-making task (houses vs. faces) and identified EEG components that classified trials as hard or as easy. Ratcliff et al. took trials from each single stimulus difficulty condition (in which nominal stimulus difficulty was constant) and applied a median split based on the amplitude of the
EEG-component. Even though nominal stimulus difficulty was
identical, estimated drift rates were lower in the trials with lower
amplitude than trials with a higher EEG amplitude. Also, the estimated across-trial variability in difficulty was smaller for the median-split groups than when estimated from all trials at once. This pattern is consistent with the hypothesis that the EEG component indexes single-trial difficulty, and that restricting difficulty (estimated by EEG) reduces estimated variability in decision difficulty (measured by the model applied to the behavioral data).

In a different approach, \citeA{van2011neural} took not the neural, but the
behavioral data as a starting point. In an attempt to leverage the power of
the single-trial regressor approach, they developed an extension of the
standard linear ballistic accumulator model \cite<LBA;>{brown2008simplest}:
the ``single-trial linear ballistic accumulator model'' (STLBA). The STLBA
provided maximum-likelihood (ML) parameter estimates for the start point of
the winning accumulator, for every trial. For many trials this value is
identical, but the ML starting points of more extreme trials offer
variability to explain corresponding variability in (neural) data. Van Maanen
et al. used this model to analyze data from subjects performing a random-dot
motion task, cued to stress either speed or accuracy. Van Maanen and
colleagues did not use the standard single-trial regression approach, where
convolved model-parameters are inserted into the GLM. Instead, the authors
first estimated the height of the BOLD-response for the individual trials,
assuming a canonical HRF \cite<see>{mumford2011deconvolving}. These
single-trial estimates were then correlated with the amount of response
caution of individual trials as estimated by the STLBA. The authors showed
that BOLD activity in the pre-SMA and the dorsal ACC correlated significantly
across trials with response caution during a speed-stressed response regime,
but not during an accuracy-stressed response regime.

\citeA{Boehm:2014cy} applied the STLBA of \citeauthor{van2011neural} to EEG
data. The authors correlated single-trial estimates of response caution against
the so-called contingent negative variation (CNV). The CNV is a buildup of
negative EEG potential that occurs when subjects are shown a cue predicting an
upcoming response stimulus. Its size was quantified by taking the mean
potential between 100msec. and 200msec. before stimulus presentation at FCz, an EEG
electrode close to pre-SMA. It turned out that the size of the CNV was linearly
related to trial-to-trial variability in response caution within the same
condition.

%% SCOTT to GILLES: The paragraph below, about Cavanagh's work, might be moved to the "joint models" section. It's about as "joint" as Turner et al.'s work, I think. I understand that a limitation is that the EEG data were pre-processed quite a lot (to get to single-trial theta-power estimates) before linking, but after that a single model was jointly fit those estimates and the behavioural data. If you agree, move this to the appropriate section.
%% GILLES to SCOTT: for me the 'joint modelling' section begins here: bayesian models where a covariance matrix is estimated between the two sources.
Even more recent approaches to linking evidence accumulation models to neural
data start with the neural signal, and use this as input to an extended
evidence accumulation model. \citeA{Cavanagh:2011kj} estimated, separately for each trial in a decision-making experiment, the power in the theta frequency band from recorded EEG signals. These single-trial estimates of theta power were then used to inform parameter estimates in an extended version of the drift diffusion model \cite<HDDM;>{Wiecki:2013ir}. This model allowed different estimates of the threshold parameter on different trials, and a covariate model to assess the association of single-trial theta power with single-trial threshold estimates. When the model was fitted to data, its parameter estimates suggested that the coefficient of the covariate was probably larger than zero, which provides evidence that response caution (measured by the threshold parameter) is related to fluctuations in theta-power in medial prefrontal cortex.% The authors then also showed that this effect reverses in patients who have Parkinson's Disease but receive deep brain stimulation in the subthalamic nucleus. When the deep brain stimulation was provided, the coefficient for the covariate model switch signs, to negative values.
%% SCOTT to GILLES: I've commented out the last two sentences above, because its not clear to me what the Parkinson's stuff means, and also how it adds to the discussion of linking hypotheses etc.

Using the same model architecture, \citeA{Frank:2015hk} extended the findings
of \citeA{Cavanagh:2011kj} to a reinforcement learning paradigm, where subjects
had to learn to choose between two stimuli with different reward probabilities,
while measuring both fMRI and EEG. A simple reinforcement learning model was
fit to the responses of the subjects, as well as a drift diffusion model, using
the approach of \citeA{Wiecki:2013ir} to yield single-trial estimates of
decision threshold. The parameter estimates were subsequently compared with
multiple neural measures, as well as the output of the reinforcement learning
model. Model comparison techniques showed that the fits of the DDM were
improved by using the value-difference between the presented choices, as
estimated by the RL model, as a covariate for the the drift rate of the DDM. In
addition, the DDM showed a better fit when the threshold parameter was included
both the EEG and fMRI measures as covariates. This showed that the threshold
parameter in the DDM model was reliably related to both cortical theta
fluctuations as measured by EEG, as well as subcortical activations as measured
by fMRI. For the fMRI data, single trial measures of BOLD activation were
estimated using a slightly modified version of the canonical hemodynamic
response function and a general linear model, applied to timecourses of
predefined regions-of-interest. This was the first neuroscience study that
relates both EEG and fMRI to a cognitive computational model at the same time.

A similar approach to that of \citeauthor{Cavanagh:2011kj} and
\citeauthor{Frank:2015hk} was developed in parallel by \citeA{Turner:2010kt}.
Also in this ``joint modelling approach'', neural measures were used in
addition to behavioral measures as input to an extended cognitive model. Turner
et al.'s approach took the covariate-based analysis further, allowing for
a general covariance matrix to link parameters of a behavioral model (the LBA
model of decision-making) with the parameters of a neural model (a GLM). This
approach supports more exploratory analyses, allowing the identification of
different mappings from cognitive parameters to neural measures by studying the
covariance matrix of the joint normal distribution; if a cognitive parameter is
related to some neural measure, the covariance parameter that links them will
be non-zero. \citeA{Turner:2010kt} showed, using the data of
\citeA{forstmann2010cortico}, that this approach can find robust correlations
between white-matter strength between pre-SMA and striatum, measured by
diffusion-weighted magnetic resonance imaging (dMRI). 

In a more recent paper, \citeA{Turner:2014tz} extended the joint modelling
aproach of \citeA{Turner:2010kt} to relate single-trial BOLD activation
estimates to single-trial parameter estimates of a drift diffusion model: the
``neural drift diffusion model''. The statistical approach was the same as used by
\citeA{Turner:2010kt}: a Bayesian generative model assumed
that both single-trial BOLD activity, as well as single-trial drift rates were
sampled from a common multivariate normal distribution. Markov chain Monte-Carlo sampling \cite<MCMC;>{kruschke2011doing,Turner:2013gb} offers the ability
to sample from the marginal posterior distributions of the model parameters,
after observing cognitive and neural data, and if neural activity is related to
cognitive parameters, the corresponding covariance estimates of the joint
normal distribution will be reliably different from zero. \citeA{Turner:2014tz} showed that
when activity in the ``default mode network'' (DMN) was large at the onset
of a trial, drift rates for that trial were lower. This
could be explained as a lack of attention, or the presence of ``task-unrelated
thoughts'', which decrease task performance.

% GILLES TO SCOTT: I'm not sure if we should drop too many of the examples. Because the problem is harder for evidence accumulation models, people have come up with quite different approaches, with a slow evolution to more robust and formal ones. I like how this part of the text summarizes the approaches that have been used in the evidence accumulation domain, with increasing order of complexity. I notice that even in the Forstmann cognitive model-based imaging lab, there is confusion about what approaches have been used and how they relate to each other. However, we might remove some duplicate approaches: maybe the Forstmann 20150 DWI-paper, the Mulder bias-paper the Boehm STLBA-in-EEG-paper 
% Aother difficulty issue for me is that the Turner and Wiecki-approaches are so similar but it seems orthogonally cited in the literature. 
% What do you think?
%% SCOTT to GILLES: I take your point. Lets leave it as you had it (although I've edited much of the text).

\section{Single model approach}
The next step in the modelling of neural and behavioral data is the development
of models that can generate and predict both neural and behavioral data at the
same time. Perhaps surprisingly, many of the approaches discussed above could
in principle do that: if a researcher knows the size of a linear relationship
between drift rate and BOLD amplitude, she can give the most-likely HRFs for
a range of drift rates. However, in the applications of quantitative predictive
modeling described above, the assumptions underlying the link are implicit and
quite simplistic; usually, the neural model is nothing more than a descriptive
measurement tool. Examples of such measurement models are the mean EEG
amplitude in a fixed time window, or the linearly estimated height of the
hemodynamic response in an fMRI signal, assuming a fixed shape across all
subjects, areas and conditions. The assumed link between the cognitive
parameter and neural measure is usually a strict linear one, with the added
assumption of Gaussian noise.

\subsection{Single model approaches in evidence accumulation}
In some work in neurophysiology, the link between neural data and cognitive
model is more explicit. The most complex models can take as input neural data
from one source, and then predict neural data from another source, as well as
behavior. \citeA{Purcell2010} identified and recorded from different clusters
of cells in the frontal eye fields (FEF) in awake macaque monkeys during
a visual search task. Some neurons in the FEF only respond to specific visual
inputs (``visual'' neurons), while other neurons respond only just before a saccade
(``motor'' neurons), and some neurons respond to both (``visuomotor'' neurons).
Considered from the perspective of an evidence accumulation model of
decision-making, the visual neurons might be interpreted as providing
a continuous, noisy, representation of decision evidence, and the motor neurons
might be interpreted as the accumulators which process that evidence.
\citeauthor{Purcell2010} did exactly this, and used the spike trains recorded
from visual and visuomotor neurons as input to the accumulators of an evidence
accumulation model. The authors showed that the model could use these inputs to
reliably predict the behavioral data of the monkeys (response proportions and
reaction time distributions). In this approach, the linking is made very
explicit, and comparisons are made at the level of \emph{distributions} over
neural and behavioral data; no single trial estimates were made.

\citeA{Purcell2010} went further and compared the predictive performance of the model using nine different architectures for evidence accumulation. These architectures differed in things like the presence or absence of leakage in the accumulation process, or mutual inhibition between accumulators. Interestingly, the response proportions and response time distributions were well explained
by many of the different model architectures, even though those architectures made very different assumptions about neural structure. Purcell et al. suggested that one way to distinguish between their nine models was to
make quantitative predictions about the properties of neurons that accumulate
evidence, the ``the motor neurons'', and then probe which model most resembles
the actual neural charachteristics. Here, the linking of cognitive model and
neural signal becomes very explicit and tight: it is assumed that the
trajectory of evidence accumulation in the model maps identically on to spiking rates in
motor neurons in FEF. The authors show that the onset of firing in motor
neurons shows a correlation with RT, growth rate shows only a very minor,
negative correlation, and baseline firing rate and ending firing rate don not
correlate with RT at all. They then show that only one class of evidence accumulation architectures showed this pattern of correlation: the ``gated'' models, which assume that accumulation does not start until the amount of incoming evidence reaches a certain threshold.


\subsection{Single model approaches in symbolic reasoning}
Simple symbolic reasoning models have been combined with functional
neuroimaging data in a single model using Hidden semi-Markov models (HSMMs).
Such models assume that, in order to perform a task, subjects move through
a discrete set of cognitive steps, or ``states'', until they finished the trial
(usually by giving a response). A HSMM can be fit to both behavioral and
neuroimaging data, where it is assumed that both are dependent measurements
of the same sequence of states. 

\citeA{Anderson:2010gi,Anderson:2011em} first applied this approach to
a dataset where students solved linear algebra problems in an MRI scanner,
where every step in solving the problem was made explicit using the task
interface. The authors showed that, given both reaction times and functional
neuroimaging data, it was possible to reliably predict in which state of
solving the linear algebra problem the subject currently is. Extending this
finding, \citeA{Anderson:2014gx} showed that, in a paradigm where the
different sub-steps of solving an algebra problem were not made explicit in
the task interface, the number of cognitive states a subject goes through in
a task can also be inferred by fitting different nested models. The authors
provide evidence that these states correspond cognitive states like ``Problem
encoding'', ``Planning, ``Solving'' and ``Responding''. One way that they do
so is inspecting the prototypical brain activation patterns corresponding to
the different cognitive states; for example, the ``respond state'' corresponds to
high activation in left motor cortex. Another crucial argument is that the
duration of these substates was specifically modulated for different
experimental conditions. For example, response time differences across difficulty levels
could almost exclusively be explained by differences in the length of the
``solving'' state. Such an effect would be very hard to investigate using only
behavioral data.

An important disadvantage of fMRI is the sluggishness of the hemodynamic
response and the low temporal resolution it therefore provides. For
experimental paradigms that (unlike algebra problems) take place on
a sub-second timescale, EEG might be a more appropriate neuroimaging method
to use in the HSMM framework, especially when one is interested in the
possible number of substates that a subject goes through solving a task.
\citeA{Borst:2015hf} combined EEG with the HSSM framework in an associative
recognition task. The authors showed that a familiarity process, an
associative retrieval process, and a decision process play a role. The study
therefore speaks in favor of a dual-process model where both processes of
recollection and familiarity are present, and against a global matching model.
Its results dissociate two main opposing theories in the literature
on associative recognition and therefore provides an excellent example of how
neuroimaging, when tightly linked to cognitive models, can help distinguish
between different cognitive theories in a way behavioral data alone cannot.


\section{Discussion}
A growing number of researchers are working towards linking formal cognitive
computational models with neuroscientific data. This linking effort is made in
vastly different fields of cognitive modelling, ranging from perceptual and
value-based decision-making to symbolic reasoning models. These models are
also linked to neural data coming from very different neuronal imaging
modalities, including single-cell recordings, EEG, and
fMRI. We described different kinds of linking that are applied in four
discrete categories that vary along a continuum of how tight and how explicit
the link between cognitive model and neural data is made. However, it is clear
that even within these categories, different analyses are applied.

The great majority of papers that have presented model-based links between
neural and behavioral data, so far, are based on a regression analysis. This
analysis tests for relationships between parameters estimated by a cognitive
model and some aspect of a neural signal -- often a parameter estimated using
a neural measurement model. In such approaches, the exact mapping from
cognitive parameter to neural signal is typically left implicit, but upon
closer inspection the link is almost always a linear relationship between
a parameter of the cognitive model and a parameter of an easy-to-use,
traditional measurement model of the neural signal. The role of the measurement
model is to reduce the raw neural data to a single number (per subject, per
condition, or per trial) which can be submitted to a standard regression
analysis. For example, in fMRI this measurement model is most often
a standard GLM used to model BOLD responses via a canonical hemodynamic
response function. The GLM allows the estimation of coefficients which index
the height of the hemodynamic response function, or the difference in height
between conditions, and it is these coefficients that are later correlated
against the parameters of a cognitive model. In EEG, the measurement model is
often nothing just the mean signal intensity in a predefined
stimulus-locked time window.

The assumptions underlying these linking functions are almost never tested,
even though there is ample evidence that they are probably often violated.
For example, a central assumption of the canonical HRF model used in fMRI is
that the HRF is identical across brain areas and subjects, but of course this
is not true \cite<e.g.,>{gonzalez2012whole}. Even more problematic is that
cognitive processes, as modeled by computational models, are typically
assumed to modulate only the \emph{amplitude} of the task-locked hemodynamic
response. Figure \ref{fig:hrfs} illustrates just how simplistic this
assumption is. Contradicting this assumption, the main finding of multiple
linking papers was a relationship between an estimated cognitive parameter
and the \emph{delay} \cite{Ho:2009da} or \emph{dispersion} of the HRF
\cite{Borst:2011jl}. We have also seen in our own data
\cite<e.g.>{forstmann2008striatum}, that cue-locked differences in the height
of the HRF across conditions are often accompanied by differences in the
onset-till-peak, as well as the dispersion of this HRF. These problems of
violated or untested assumptions are not unique to fMRI measurement. For
example, a frequent assumption in analyses of single-cell recordings is
independence between recordings taken over different trials, cells, or
conditions, but this is often not true \cite{Aarts:2014io}.

The more advanced linking approaches we have reviewed, particularly the
quantitative predictive approach and the single-model approach, enable
future work to focus on more complex relationships and formally incorporate
them in their quantitative models. A useful strategy might be to steer away
from standard approaches where many thousands of voxels are all regressed
against the same cognitive parameters, using standard scripts, and, instead
focus on confirmatory, hypothesis-driven analyses. Single model linking
approaches lend themselves very well to confirmatory approaches, as they
allow -- indeed, require -- the researcher to build into the model explicit
statements about exactly which and how cognitive parameters are related to
the neural data. For example, a single model may do nothing more than link
the BOLD signal from a predefined ananatomically-informed region-of-interest
via a linear relationship between a cognitive model parameter and the height
of a hemodynamic response function. Even though this limited example goes no
further than many standard approaches, it makes all the assumptions
explicit, and testable -- e.g. it would be easy to compare this model
against one where the link function was sigmoidal, or was on the dispersion
rather than height of the hemodynamic response function. A further advantage
of reducing the tens-of-thousands single-voxel-signals to a limited set of
anatomicaly-defined signals, is that more complex models, such as those that
allow for shifts in onsets-to-peak and dispersion of the task-locked BOLD
signal, also become computationally tractable. Bayesian probabilistic
graphical models make it easy to link such models of neural activity to
models of cognition, as both can be formalized with a set of quanatiative
parameters and a likelihood function \cite{Turner:2010kt}. Such approaches
would also fit well into the recent trend in functional neuroimaging research
of a lesser focus on voxelwise, univariate analyses, and an emphasis on
spatioptemporal patterns in corresponding anatomical regions in the brain
\cite{Lohmann:2013kc}.


\begin{figure}
  %\centering
  \includegraphics[width=.8\textwidth]{hrfs.png}
  \caption{Standard canonical HRF model of BOLD-activity. In regression-based approaches, it is assumed that the only way the BOLD-response is modulated is that the canonical shape is multiplied by some factor $\beta$. \label{fig:hrfs}}
\end{figure}

Even if these quantitative links between model parameters and neural signals
are made more explicit, researchers must still remain very careful and
circumspect in the conclusions they draw from any of these statistical links
they find between the two. Clearly, computational models that were developed in
cognitive psychology are there to explain \emph{cognition}. At best, they give
a formalization of the kind of cognitive processes the human brain can perform,
how they differ in different circumstances, and formalize differences in
cognition across subjects. They describe the algorithms that take place in the
brain and which quantities it therefore has to compute, but these models remain
agnostic about the precise implementation of these algorithms, the level of
neural signals \cite{Moore:1956vr,Schall:2004ei}. 

When a brain region is identified that shows a correlation between neural
signal and a cognitive parameter, this area may be involved in computing the
quantity that corresponds to that parameter. Nevertheless, it is
still very unlikely that such a cognitive parameter is a cognitive process in
itself, and that is has a simple one-to-one, cognitive-process-to-brain-area
mapping  \cite{OReilly:2011ca}. This is because many other hypotheses can
be generated which are consistent with the observed link, but differ in the
mechanics that explain the link (e.g., perhaps the identified region simply
relays or mirrors the signal of interest). Finding relationships between
cognitive models and neural measurements is just a first step: an excellent
stepping stone toward more detailed neurocomputational models
\cite<e.g.,>{bogacz2010neural}.

\subsection{When is a linking approach not good enough?}

It is difficult to avoid the appearance of value judgments in our proposed
continuum of loose-to-tight linking. However, we would like to stress that it
is not true that tighter linking is always preferable to looser linking. The
ideal point on the continuum depends on many things, but most importantly on
the status of the formal quantitative models in the research paradigm of
interest. Tighter linking approaches are only feasible when there exist
well-understood and settled quantitative models of both the behavioral
data and the neural data in the paradigm of interest. Even more, these models
must be computationally or analytically tractable, for any level of linking,
and they should preferably specified at the level of distributions and
likelihood functions for the tightest possible linking. Where these assumptions
are not met, linking approaches beyond the simplest qualitative structural or
qualitative predictive are not likely to succeed (e.g., this was the case for
models of perceptual decision-making in the late 1990s and early 2000s). Conversely, in areas
for which tractable quantitative models do exist, it is incumbent on
researchers to strive for the tightest possible linking approaches (e.g., it is
no longer reasonable for perceptual decision-making models to be linked to
neural data using qualitative approaches).


\section{Conclusions}

Linking formal cognitive models to neural data can improve our understanding of
how the brain functions. However, the precise technical details and
philosophical approach to tackling this linking problem can vary greatly. We
have focused on one important attribute of linking: the degree of tight,
quantitative, and explicit hypothesizing. In each field of cognitive research,
the earliest approaches to linking behavioral and neural data are typically
qualitative. As knowledge is accumulated, and as formal models become more
settled, linking approaches become more quantitative and more explicit. Even in
that case, however, model parameters are most frequently estimated separately
for the behavioral and neural models, and the two models are brought together only
at the very end, in a simple linear regression on model parameters. Future work
should explicitly model how differences in cognitive parameters modulate
differences in the signal in the neural domain, thereby acknowledging the
richness of the data in the neural domain and exploit more than the most common
parameter of the most simple measurement model in that domain.

\bibliography{sbrefs,library_gilles}

\end{document}  
